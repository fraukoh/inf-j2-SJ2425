% Options for packages loaded elsewhere
\PassOptionsToPackage{unicode}{hyperref}
\PassOptionsToPackage{hyphens}{url}
\PassOptionsToPackage{dvipsnames,svgnames,x11names}{xcolor}
%
\documentclass[
  letterpaper,
  DIV=11,
  numbers=noendperiod]{scrartcl}

\usepackage{amsmath,amssymb}
\usepackage{iftex}
\ifPDFTeX
  \usepackage[T1]{fontenc}
  \usepackage[utf8]{inputenc}
  \usepackage{textcomp} % provide euro and other symbols
\else % if luatex or xetex
  \usepackage{unicode-math}
  \defaultfontfeatures{Scale=MatchLowercase}
  \defaultfontfeatures[\rmfamily]{Ligatures=TeX,Scale=1}
\fi
\usepackage{lmodern}
\ifPDFTeX\else  
    % xetex/luatex font selection
\fi
% Use upquote if available, for straight quotes in verbatim environments
\IfFileExists{upquote.sty}{\usepackage{upquote}}{}
\IfFileExists{microtype.sty}{% use microtype if available
  \usepackage[]{microtype}
  \UseMicrotypeSet[protrusion]{basicmath} % disable protrusion for tt fonts
}{}
\makeatletter
\@ifundefined{KOMAClassName}{% if non-KOMA class
  \IfFileExists{parskip.sty}{%
    \usepackage{parskip}
  }{% else
    \setlength{\parindent}{0pt}
    \setlength{\parskip}{6pt plus 2pt minus 1pt}}
}{% if KOMA class
  \KOMAoptions{parskip=half}}
\makeatother
\usepackage{xcolor}
\setlength{\emergencystretch}{3em} % prevent overfull lines
\setcounter{secnumdepth}{-\maxdimen} % remove section numbering
% Make \paragraph and \subparagraph free-standing
\ifx\paragraph\undefined\else
  \let\oldparagraph\paragraph
  \renewcommand{\paragraph}[1]{\oldparagraph{#1}\mbox{}}
\fi
\ifx\subparagraph\undefined\else
  \let\oldsubparagraph\subparagraph
  \renewcommand{\subparagraph}[1]{\oldsubparagraph{#1}\mbox{}}
\fi

\usepackage{color}
\usepackage{fancyvrb}
\newcommand{\VerbBar}{|}
\newcommand{\VERB}{\Verb[commandchars=\\\{\}]}
\DefineVerbatimEnvironment{Highlighting}{Verbatim}{commandchars=\\\{\}}
% Add ',fontsize=\small' for more characters per line
\usepackage{framed}
\definecolor{shadecolor}{RGB}{241,243,245}
\newenvironment{Shaded}{\begin{snugshade}}{\end{snugshade}}
\newcommand{\AlertTok}[1]{\textcolor[rgb]{0.68,0.00,0.00}{#1}}
\newcommand{\AnnotationTok}[1]{\textcolor[rgb]{0.37,0.37,0.37}{#1}}
\newcommand{\AttributeTok}[1]{\textcolor[rgb]{0.40,0.45,0.13}{#1}}
\newcommand{\BaseNTok}[1]{\textcolor[rgb]{0.68,0.00,0.00}{#1}}
\newcommand{\BuiltInTok}[1]{\textcolor[rgb]{0.00,0.23,0.31}{#1}}
\newcommand{\CharTok}[1]{\textcolor[rgb]{0.13,0.47,0.30}{#1}}
\newcommand{\CommentTok}[1]{\textcolor[rgb]{0.37,0.37,0.37}{#1}}
\newcommand{\CommentVarTok}[1]{\textcolor[rgb]{0.37,0.37,0.37}{\textit{#1}}}
\newcommand{\ConstantTok}[1]{\textcolor[rgb]{0.56,0.35,0.01}{#1}}
\newcommand{\ControlFlowTok}[1]{\textcolor[rgb]{0.00,0.23,0.31}{#1}}
\newcommand{\DataTypeTok}[1]{\textcolor[rgb]{0.68,0.00,0.00}{#1}}
\newcommand{\DecValTok}[1]{\textcolor[rgb]{0.68,0.00,0.00}{#1}}
\newcommand{\DocumentationTok}[1]{\textcolor[rgb]{0.37,0.37,0.37}{\textit{#1}}}
\newcommand{\ErrorTok}[1]{\textcolor[rgb]{0.68,0.00,0.00}{#1}}
\newcommand{\ExtensionTok}[1]{\textcolor[rgb]{0.00,0.23,0.31}{#1}}
\newcommand{\FloatTok}[1]{\textcolor[rgb]{0.68,0.00,0.00}{#1}}
\newcommand{\FunctionTok}[1]{\textcolor[rgb]{0.28,0.35,0.67}{#1}}
\newcommand{\ImportTok}[1]{\textcolor[rgb]{0.00,0.46,0.62}{#1}}
\newcommand{\InformationTok}[1]{\textcolor[rgb]{0.37,0.37,0.37}{#1}}
\newcommand{\KeywordTok}[1]{\textcolor[rgb]{0.00,0.23,0.31}{#1}}
\newcommand{\NormalTok}[1]{\textcolor[rgb]{0.00,0.23,0.31}{#1}}
\newcommand{\OperatorTok}[1]{\textcolor[rgb]{0.37,0.37,0.37}{#1}}
\newcommand{\OtherTok}[1]{\textcolor[rgb]{0.00,0.23,0.31}{#1}}
\newcommand{\PreprocessorTok}[1]{\textcolor[rgb]{0.68,0.00,0.00}{#1}}
\newcommand{\RegionMarkerTok}[1]{\textcolor[rgb]{0.00,0.23,0.31}{#1}}
\newcommand{\SpecialCharTok}[1]{\textcolor[rgb]{0.37,0.37,0.37}{#1}}
\newcommand{\SpecialStringTok}[1]{\textcolor[rgb]{0.13,0.47,0.30}{#1}}
\newcommand{\StringTok}[1]{\textcolor[rgb]{0.13,0.47,0.30}{#1}}
\newcommand{\VariableTok}[1]{\textcolor[rgb]{0.07,0.07,0.07}{#1}}
\newcommand{\VerbatimStringTok}[1]{\textcolor[rgb]{0.13,0.47,0.30}{#1}}
\newcommand{\WarningTok}[1]{\textcolor[rgb]{0.37,0.37,0.37}{\textit{#1}}}

\providecommand{\tightlist}{%
  \setlength{\itemsep}{0pt}\setlength{\parskip}{0pt}}\usepackage{longtable,booktabs,array}
\usepackage{calc} % for calculating minipage widths
% Correct order of tables after \paragraph or \subparagraph
\usepackage{etoolbox}
\makeatletter
\patchcmd\longtable{\par}{\if@noskipsec\mbox{}\fi\par}{}{}
\makeatother
% Allow footnotes in longtable head/foot
\IfFileExists{footnotehyper.sty}{\usepackage{footnotehyper}}{\usepackage{footnote}}
\makesavenoteenv{longtable}
\usepackage{graphicx}
\makeatletter
\def\maxwidth{\ifdim\Gin@nat@width>\linewidth\linewidth\else\Gin@nat@width\fi}
\def\maxheight{\ifdim\Gin@nat@height>\textheight\textheight\else\Gin@nat@height\fi}
\makeatother
% Scale images if necessary, so that they will not overflow the page
% margins by default, and it is still possible to overwrite the defaults
% using explicit options in \includegraphics[width, height, ...]{}
\setkeys{Gin}{width=\maxwidth,height=\maxheight,keepaspectratio}
% Set default figure placement to htbp
\makeatletter
\def\fps@figure{htbp}
\makeatother

\KOMAoption{captions}{tableheading}
\makeatletter
\@ifpackageloaded{caption}{}{\usepackage{caption}}
\AtBeginDocument{%
\ifdefined\contentsname
  \renewcommand*\contentsname{Table of contents}
\else
  \newcommand\contentsname{Table of contents}
\fi
\ifdefined\listfigurename
  \renewcommand*\listfigurename{List of Figures}
\else
  \newcommand\listfigurename{List of Figures}
\fi
\ifdefined\listtablename
  \renewcommand*\listtablename{List of Tables}
\else
  \newcommand\listtablename{List of Tables}
\fi
\ifdefined\figurename
  \renewcommand*\figurename{Figure}
\else
  \newcommand\figurename{Figure}
\fi
\ifdefined\tablename
  \renewcommand*\tablename{Table}
\else
  \newcommand\tablename{Table}
\fi
}
\@ifpackageloaded{float}{}{\usepackage{float}}
\floatstyle{ruled}
\@ifundefined{c@chapter}{\newfloat{codelisting}{h}{lop}}{\newfloat{codelisting}{h}{lop}[chapter]}
\floatname{codelisting}{Listing}
\newcommand*\listoflistings{\listof{codelisting}{List of Listings}}
\makeatother
\makeatletter
\makeatother
\makeatletter
\@ifpackageloaded{caption}{}{\usepackage{caption}}
\@ifpackageloaded{subcaption}{}{\usepackage{subcaption}}
\makeatother
\ifLuaTeX
  \usepackage{selnolig}  % disable illegal ligatures
\fi
\usepackage{bookmark}

\IfFileExists{xurl.sty}{\usepackage{xurl}}{} % add URL line breaks if available
\urlstyle{same} % disable monospaced font for URLs
\hypersetup{
  colorlinks=true,
  linkcolor={blue},
  filecolor={Maroon},
  citecolor={Blue},
  urlcolor={Blue},
  pdfcreator={LaTeX via pandoc}}

\author{}
\date{}

\begin{document}

\section{Aufgabe 1 {[}3 P{]}:}\label{aufgabe-1-3-p}

Erstellen Sie ein Array mit den Lieblingsfächern
\texttt{{[}"Mathematik",\ "Biologie",\ "Informatik",\ "Chemie",\ "Geschichte"{]}}
und geben Sie jedes Fach mit dem Text ``Lieblingsfach:'' davor aus.

\begin{Shaded}
\begin{Highlighting}[]
\NormalTok{lieblingsfaecher }\OperatorTok{=}\NormalTok{ [}\StringTok{"Mathematik"}\NormalTok{, }\StringTok{"Biologie"}\NormalTok{, }\StringTok{"Informatik"}\NormalTok{, }\StringTok{"Chemie"}\NormalTok{, }\StringTok{"Geschichte"}\NormalTok{]}
\ControlFlowTok{for}\NormalTok{ i }\KeywordTok{in} \BuiltInTok{range}\NormalTok{(}\BuiltInTok{len}\NormalTok{(lieblingsfaecher)):}
    \BuiltInTok{print}\NormalTok{(}\StringTok{"Lieblingsfach: "} \OperatorTok{+}\NormalTok{ lieblingsfaecher[i])}
\end{Highlighting}
\end{Shaded}

\begin{verbatim}
Lieblingsfach: Mathematik
Lieblingsfach: Biologie
Lieblingsfach: Informatik
Lieblingsfach: Chemie
Lieblingsfach: Geschichte
\end{verbatim}

\section{Aufgabe 2 {[}4 P{]}:}\label{aufgabe-2-4-p}

Erstellen Sie ein Array mit den monatlichen Ausgaben
\texttt{{[}200,\ 250,\ 300,\ 150,\ 400,\ 350,\ 280,\ 320,\ 370,\ 410,\ 380,\ 450{]}}
und berechnen Sie die Gesamtausgaben der \textbf{ersten Hälfte} des
Jahres.

Hinweis: Verwenden Sie dafür eine Schleife

\begin{Shaded}
\begin{Highlighting}[]
\NormalTok{ausgaben }\OperatorTok{=}\NormalTok{ [}\DecValTok{200}\NormalTok{, }\DecValTok{250}\NormalTok{, }\DecValTok{300}\NormalTok{, }\DecValTok{150}\NormalTok{, }\DecValTok{400}\NormalTok{, }\DecValTok{350}\NormalTok{, }\DecValTok{280}\NormalTok{, }\DecValTok{320}\NormalTok{, }\DecValTok{370}\NormalTok{, }\DecValTok{410}\NormalTok{, }\DecValTok{380}\NormalTok{, }\DecValTok{450}\NormalTok{]}
\NormalTok{erste\_haelfte\_ausgaben }\OperatorTok{=} \DecValTok{0}
\ControlFlowTok{for}\NormalTok{ i }\KeywordTok{in} \BuiltInTok{range}\NormalTok{(}\DecValTok{6}\NormalTok{):}
\NormalTok{    erste\_haelfte\_ausgaben }\OperatorTok{=}\NormalTok{ erste\_haelfte\_ausgaben }\OperatorTok{+}\NormalTok{ ausgaben[i]}
\BuiltInTok{print}\NormalTok{(}\StringTok{"Die Gesamtausgaben der ersten Hälfte des Jahres sind:"}\NormalTok{, erste\_haelfte\_ausgaben)}
\end{Highlighting}
\end{Shaded}

\begin{verbatim}
Die Gesamtausgaben der ersten Hälfte des Jahres sind: 1650
\end{verbatim}

\section{Aufgabe 3 {[}4 P{]}:}\label{aufgabe-3-4-p}

Erstellen Sie ein Array mit den Projekten des Schuljahres
\texttt{{[}"Kunstausstellung",\ "Sporttag",\ "Wissenschaftsmesse"{]}}
und fügen Sie das Projekt ``Schulwebsite'' an der ersten Stelle (=ganz
am Anfang) hinzu.

\begin{Shaded}
\begin{Highlighting}[]
\NormalTok{projekte }\OperatorTok{=}\NormalTok{ [}\StringTok{"Kunstausstellung"}\NormalTok{, }\StringTok{"Sporttag"}\NormalTok{, }\StringTok{"Wissenschaftsmesse"}\NormalTok{]}
\NormalTok{projekte.append(}\StringTok{"Schulwebsite"}\NormalTok{)}
\ControlFlowTok{for}\NormalTok{ i }\KeywordTok{in} \BuiltInTok{range}\NormalTok{(}\BuiltInTok{len}\NormalTok{(projekte)}\OperatorTok{{-}}\DecValTok{1}\NormalTok{, }\DecValTok{0}\NormalTok{, }\OperatorTok{{-}}\DecValTok{1}\NormalTok{):}
\NormalTok{    zwischengespeichert }\OperatorTok{=}\NormalTok{ projekte[i]}
\NormalTok{    projekte[i] }\OperatorTok{=}\NormalTok{ projekte[i}\OperatorTok{{-}}\DecValTok{1}\NormalTok{]}
\NormalTok{    projekte[i}\OperatorTok{{-}}\DecValTok{1}\NormalTok{] }\OperatorTok{=}\NormalTok{ zwischengespeichert}
\BuiltInTok{print}\NormalTok{(}\StringTok{"Aktualisierte Projekte:"}\NormalTok{, projekte)}
\end{Highlighting}
\end{Shaded}

\begin{verbatim}
Aktualisierte Projekte: ['Schulwebsite', 'Kunstausstellung', 'Sporttag', 'Wissenschaftsmesse']
\end{verbatim}

\section{Aufgabe 4 {[}4 P{]}:}\label{aufgabe-4-4-p}

Gegeben sind zwei Arrays:

\texttt{kunden\ =\ {[}"Max\ Mustermann",\ "Erika\ Musterfrau",\ "Hans\ Müller",\ "Anna\ Schmidt",\ "Peter\ Meier"{]}}\strut \\
\texttt{kundennummern\ =\ {[}"001",\ "002",\ "003",\ "004",\ "005"{]}}

Schreiben Sie ein Programm, das bei Eingabe einer oben genannten
Kundennummer den jeweiligen Kundennamen anzeigt.

\begin{Shaded}
\begin{Highlighting}[]
\NormalTok{kunden }\OperatorTok{=}\NormalTok{ [}\StringTok{"Max Mustermann"}\NormalTok{, }\StringTok{"Erika Musterfrau"}\NormalTok{, }\StringTok{"Hans Müller"}\NormalTok{, }\StringTok{"Anna Schmidt"}\NormalTok{, }\StringTok{"Peter Meier"}\NormalTok{]}
\NormalTok{kundennummern }\OperatorTok{=}\NormalTok{ [}\StringTok{"001"}\NormalTok{, }\StringTok{"002"}\NormalTok{, }\StringTok{"003"}\NormalTok{, }\StringTok{"004"}\NormalTok{, }\StringTok{"005"}\NormalTok{]}

\NormalTok{eingabe\_nummer }\OperatorTok{=} \BuiltInTok{input}\NormalTok{(}\StringTok{"Bitte geben Sie die Kundennummer ein: "}\NormalTok{)}
\ControlFlowTok{for}\NormalTok{ i }\KeywordTok{in} \BuiltInTok{range}\NormalTok{(}\BuiltInTok{len}\NormalTok{(kundennummern)):}
    \ControlFlowTok{if}\NormalTok{ eingabe\_nummer }\OperatorTok{==}\NormalTok{ kundennummern[i]:}
        \BuiltInTok{print}\NormalTok{(}\StringTok{"Der Kunde mit der Kundennummer"}\NormalTok{, eingabe\_nummer, }\StringTok{"ist:"}\NormalTok{, kunden[i])}
\end{Highlighting}
\end{Shaded}

\begin{verbatim}
Der Kunde mit der Kundennummer 003 ist: Hans Müller
\end{verbatim}

\section{Aufgabe 5 {[}6 P{]}:}\label{aufgabe-5-6-p}

Erstellen Sie ein Array mit den Namen von Autoren
\texttt{{[}"Hermann\ Hesse",\ "Franz\ Kafka",\ "Thomas\ Mann",\ "Bertolt\ Brecht"{]}}
und sortieren Sie sie alphabetisch mit BubbleSort.

\begin{Shaded}
\begin{Highlighting}[]
\NormalTok{autoren }\OperatorTok{=}\NormalTok{ [}\StringTok{"Hermann Hesse"}\NormalTok{, }\StringTok{"Franz Kafka"}\NormalTok{, }\StringTok{"Thomas Mann"}\NormalTok{, }\StringTok{"Bertolt Brecht"}\NormalTok{]}
\NormalTok{n }\OperatorTok{=} \BuiltInTok{len}\NormalTok{(autoren)}

\CommentTok{\# BubbleSort Algorithmus}
\ControlFlowTok{for}\NormalTok{ i }\KeywordTok{in} \BuiltInTok{range}\NormalTok{(n):}
    \ControlFlowTok{for}\NormalTok{ j }\KeywordTok{in} \BuiltInTok{range}\NormalTok{(}\DecValTok{0}\NormalTok{, n}\OperatorTok{{-}}\NormalTok{i}\OperatorTok{{-}}\DecValTok{1}\NormalTok{):}
        \ControlFlowTok{if}\NormalTok{ autoren[j] }\OperatorTok{\textgreater{}}\NormalTok{ autoren[j}\OperatorTok{+}\DecValTok{1}\NormalTok{]:}
\NormalTok{            zwischengespeichert }\OperatorTok{=}\NormalTok{ autoren[j]}
\NormalTok{            autoren[j] }\OperatorTok{=}\NormalTok{ autoren[j}\OperatorTok{+}\DecValTok{1}\NormalTok{]}
\NormalTok{            autoren[j}\OperatorTok{+}\DecValTok{1}\NormalTok{] }\OperatorTok{=}\NormalTok{ zwischengespeichert}

\BuiltInTok{print}\NormalTok{(}\StringTok{"Sortierte Autoren:"}\NormalTok{, autoren)}
\end{Highlighting}
\end{Shaded}

\begin{verbatim}
Sortierte Autoren: ['Bertolt Brecht', 'Franz Kafka', 'Hermann Hesse', 'Thomas Mann']
\end{verbatim}



\end{document}
